\documentclass[12pt]{article}

\usepackage[czech]{babel}
\usepackage[margin=1in]{geometry}
\usepackage[IL2]{fontenc}
\usepackage{amsmath}

\begin{document}
\begin{center}
  Geometrie a aplikovaná matematika

  (22.5.2024, 11:00, 60 minut)  
\end{center}

\begin{enumerate}

\item Derivace.
  \begin{enumerate}
  \item Napište definici derivace a vysvětlete jednotlivé komponenty této rovnice. Co vyjadřuje podíl v této definici? Proč je tam limita? 
\item V jistých speciálních případech je možno limitu v definici derivace vynechat a použít jenom podíl. Kdy?
\item Funkce $P(t)$ udává výšku premiérova platu jako funkci času. Co bude udávat derivace $\frac{\mathrm dP}{\mathrm d t}$ této funkce a jakou bude mít jednotku?
\item Může se někdy stát, že derivace bude záporná? Pokud ano, uveďte kdy nebo uveďte nějaký případ, kde je zápornost evidentní. Pokud ne, vysvětlete proč. 
\end{enumerate}
  
\item Divergence.
  \begin{enumerate}
  \item Napište definici a fyzikální význam divergence. 
  \item Zformulujte difuzní rovnici nebo rovnici kontinuity a ke
    každému členu této rovnice napište, co fyzikálně představuje.
  \item Jaké úlohy modelujeme pomocí difuzní rovnice nebo rovnice kontinuity? Napište alespoň jednu typickou.
\end{enumerate}
  

\item Matice.
  \begin{enumerate}
  \item Vysvětlete, jak je definováno maticové násobení a jakou roli
    hraje v Fourierově zákoně nebo jiném vhodném konstitučním
    zákoně. Zformulujte i tento zákon a vysvětlete jednotlivé
    veličiny, které v tomto zákoně figurují.
  \item Co jsou to vlastní směry (vlastní vektory) matice a čím jsou
    zajímavé z hlediska materiálových vlastností?
  \item V konstitučních zákonech můžeme někdy místo maticové veličiny
    uvažovat veličinu skalární. Kdy?
\end{enumerate}

  
\item Diferenciální rovnice
  \begin{enumerate}
  \item Rychlost poklesu teploty horké kávy je úměrná rozdílu teploty
    kávy a okolí. Napište diferenciální rovnici modelující tento děj.
  \item V předchozím modelu figuruje nějaká konstanta úměrnosti. Dejme
    tomu, že káva chladne jednou v plechovém hrnečku a jednou v
    polystyrenovém kelímku. Pro který případ je hodnota konstanty
    úměrnosti větší? Odpověď vysvětlete.
  \item Jakou má konstanta úměrnosti fyzikální jednotku?
\end{enumerate}


  
\end{enumerate}

\end{document}

\documentclass[12pt]{article}

\usepackage[czech]{babel}
\usepackage[margin=1in]{geometry}
\usepackage[IL2]{fontenc}
\usepackage{amsmath}

\begin{document}
\begin{center}
  Geometrie a aplikovaná matematika

  (60 minut)  
\end{center}

\begin{enumerate}

\item Rovnice vedení tepla.
  \begin{enumerate}
  \item Napište rovnici vedení tepla v jednodimenzionálním případě a
  vysvětlete jednotlivé komponenty. 
\item Vysvětlete, která část rovnice
  představuje spád teploty, tok tepla, úbytek toku tepla, co vyjadřují
  jednotlivé parciální derivace apod.
\item Na co se tato rovnice používá?
\item Pokuste se do rovnice vedení tepla přidat spotřebič tepla
\end{enumerate}
  
\item Difuzní rovnice.
  \begin{enumerate}
  \item Napište dvou nebo trojrozměrnou difuzní rovnici a podobně jako
    v předchozím případě vysvětlete jednotlivé komponenty, které mají
    nějaký fyzikální význam. 
  \item Zpravidla se snažíme, aby difuzní matice
    byla co nejednodušší. Jak je možné tohoto dosáhnout a jak bude
    tento jednoduchý tvar vypadat?
  \item Jak se bude lišit difuzní rovnice se zdroji a bez zdrojů?
\end{enumerate}
  

\item Matice.
  \begin{enumerate}
  \item Vysvětlete, jak je definováno maticové násobení a jakou roli
    hraje v Hookově zákoně. Vysvětlete i všechny veličiny figurující v
    Hookově zákoně.
  \item Uveďte příklad materiálu nebo úlohy, u kterého při modelování
    podle Hookova zákona musíme použít matice a příklad materiálu nebo
    úlohy, u kterého matice použít nemusíme.
  \item Popište stručně nějaké využití maticového součinu. Jiné než
    Hookův zákon nebo difuze.
\end{enumerate}


  
\item Linearizace funkce jedné proměnné:
  \begin{enumerate}
  \item vzorec, 
  \item vysvětlení jednotlivých částí vzorce pro porozumění,
    jak je sestaven,
  \item příklad použití.
  \item Může se linearizace za nějakých podmínek redukovat na přímou
    úměrnost? Pokud ano, napište buď obecně kdy, nebo nějaký konkrétní
    případ, kdy k tomuto dojde.
\end{enumerate}


  
\end{enumerate}

\end{document}
